\documentclass[12pt,a4paper,UTF8]{ctexbook}
\usepackage[top=2cm, bottom=3cm, left=2.5cm, right=2.5cm]{geometry} % 页边距
\usepackage{amsmath,amssymb}    % 数学公式
\usepackage{graphicx}           % 图片插入
\usepackage{booktabs}           % 专业表格
\usepackage{multirow}            % 表格跨行
\usepackage{listings}
\usepackage{bussproofs}
\usepackage{tikz}
\usepackage{multirow}   % 合并行
\usepackage{hhline}     % 绘制自定义水平线
\usepackage{array}      % 调整列格式
\usepackage[colorlinks=true]{hyperref} % 超链接
\usepackage{amssymb}
\usepackage{array}
\usepackage[dvipsnames]{xcolor}
\usepackage{mathpartir}
\usepackage{amsthm}
\usepackage{longtable}
\usepackage[utf8]{inputenc}
\usepackage[english]{babel}
\usepackage{algorithm}
\usepackage{algpseudocode}
\usepackage{algorithmicx}
\theoremstyle{plain}
\newtheorem{theorem}{\indent Theorem}[chapter]
\newtheorem{lemma}[theorem]{\indent Lemma}
\newtheorem{conjecture}[theorem]{\indent Conjecture}
\newtheorem{note}[theorem]{\indent Notation}
\newtheorem{proposition}[theorem]{\indent Proposition}
\newtheorem{corollary}[theorem]{\indent 推论}
\newtheorem{definition}{\indent Definition}[chapter]
\newtheorem{example}{\indent Example}[chapter]
\newtheorem{remark}{\indent Remark}[chapter]
\newenvironment{solution}{\begin{proof}[\indent\bf Solution]}{\end{proof}}
\renewcommand{\proofname}{\indent\bf Proof}
\lstset{
 language=sql, % 设置语言
 basicstyle=\ttfamily, % 设置字体族
 breaklines=true, % 自动换行
 keywordstyle=\bfseries\color{Aquamarine}, % 设置关键字为粗体,颜色为 NavyBlue
 morekeywords={}, % 设置更多的关键字,用逗号分隔
 emph={self}, % 指定强调词,如果有多个,用逗号隔开
    emphstyle=\bfseries\color{Rhodamine}, % 强调词样式设置
    commentstyle=\itshape\color{black!50!white}, % 设置注释样式,斜体,浅灰色
    stringstyle=\bfseries\color{PineGreen!90!black}, % 设置字符串样式
    columns=flexible,
    numbers=left, % 显示行号在左边
    numbersep=2em, % 设置行号的具体位置
    numberstyle=\footnotesize, % 缩小行号
    frame=single, % 边框
    framesep=1em % 设置代码与边框的距离
}
\def\ctime{\bf{TIME}}
\def\cspace{\bf{SPACE}}
\def\ncone{\bf{NC^1}}
\def\logsp{\bf{L}}
\newcommand{\floor}[1]{\lfloor #1 \rfloor}

\title{\bf{Introduction to Database (H)}}
\author{Yanqiao Chen, SUSTech, Shenzhen\\
        Instructor: Shiqi Yu}

\begin{document}
\setlength{\arrayrulewidth}{0.4pt}
\newcommand{\thickline}{\noalign{\hrule height 2pt}}

\maketitle
\tableofcontents

\chapter{数据库引论} 

\section{数据库历史}
\subsection{为什么需要数据库?} 
我们需要管理数据,查询、修改数据,同时需要管理多个用户读写的并发问题,因此需要开发一套合适的数据库
系统进行数据管理。 

\subsection{关系数据库} 
Edgar Codd在1970年代提出的关系模型,为后续关系数据库的开发奠定了基础。关系数据库 
的最小单位是“行”,再上面还有页(Page)、表(Table)。关系数据库中存在很多操作,一些基本操作是:
\begin{itemize} 
    \item select($\sigma_{P}(tag)$): 取出符合输入关系满足谓词的行;
    \item project($\Pi_{tag1, tag2}(tag)$): 对输入关系所有的行输出指定的属性。从输出中取出重复元组。
    \item join: 输出两个输入关系中元组的并。
    \item product: 从两个输入关系中输出所有元组对;
    \item concat: 从两个输入关系输出这样的元组对:它们在具有相同名字的所有属性上取值相同。
\end{itemize} 
数据库中,每一行本质上是没有顺序的,也就是一个集合(无序、不重合)。为了区分每一个数据,我们
为每一个数据定义一个键值(Key): 
\begin{definition}[键值] 
    键值是识别一条数据的唯一表示,可以是对于某些数据的哈希值,或者是按照初次写入顺序的标号。总而言之,
    键值必须要是独一无二的。
\end{definition}

\begin{remark}[Primary Key]
    我们可能使用很多键值,但是我们可以选出一个基本的键值作为“Primary Key”。一般来说,我们使用整数来进行标号。
\end{remark} 

对于很多数据的写法,我们需要对数据进行规范化,方便进行查询和修改。
\begin{definition}[规范化]
    对于指示同一个属性的标签,标签可能有很多种写法。同样,数据的标签可能根本不存在。我们在设计相关的
    标签的时候要考虑到这一点。
\end{definition} 
例如,汉族人通常有姓和名,但是其它民族的人可能没有姓氏,外国人可能存在姓名倒置,可能存在昵称等等。这里,
我们需要设计一些机制对标签进行规范化。

\begin{example}
    例如,可以将另一个表的索引作为标签,从而化简一张表上的数据储存。
\end{example}

\begin{example}
    还可以,我们将不同类型的数据分开存,通过第三张表记录它们的关系。这就是关系数据库的基本思想。
\end{example}


表的关系可以视为一张无向图。

\subsection{范式} 
在设计数据库的过程中,我们往往遵循一些工程范式。
\begin{itemize} 
    \item Simple Attribute(属性原子化);
    \item Attribute depend on Full Key(属性描述键);
    \item Non-Key Attribute do not depend on each other(属性不相关);
\end{itemize} 


\chapter{SQL语句}
SQL语言在不同的数据库中都会有些许不同,这被称为SQL方言。因此,我们在使用对应的数据库的时候,都需要进行
一定程度上的DIY。SQL是大小写不敏感的,但是尽量使用大写标识关键字,小写标识标识符(注意,数据大小写有别)。
\section{C/S结构}
我们所安装的Postgre是一个数据库服务器(Server),我们需要通过客户端(Client)访问服务器。
首先,我们通过服务器与服务器端口建立连接,通过用户名和密码登入服务器。然后,我们通过客户端的
图形界面输入SQL语言与服务器进行交互。我们的执行顺序如下:
\begin{itemize} 
    \item 客户端输入SQL查询(或者数据处理)命令
    \item 服务器接受SQL命令,挂起事务,检查权限,处理数据,发回客户端;
    \item 客户端接收服务器的数据;
\end{itemize}

\section{SQL语句标准} 
每一个SQL语句由一个命令序列,一个命令是一个记号序列(记号可以是一个关键字、一个标识符、一个带引号的标识符、一个literal或者常量、一个特殊字符)并且由一个分号终结。


\section{DDL语句} 

\subsection{CREATE}
CREATE语句用于创建表、行等等。在创建表的过程中,每一列的属性具有特定的类型,例如整数、
字符串等等。SQL的数据类型包含
\begin{itemize} 
  \item \textbf{Text}: \textbf{char(length)} ,\textbf{char}默认长度为1, \textbf{varchar}(可变长度,不超过最长长度), \textbf{clob}, \textbf{varchar2}(Oracle方言);
  \item \textbf{Number}: \textbf{int}, \textbf{float}(浮点数的数字标识范围大于其位宽,易出错), \textbf{numeric(精确位, 小数位)}(精确数值类型);
  \item \textbf{Date}: \textbf{date}(Oracle方言), \textbf{datetime}, \textbf{timestamp};
  \item \textbf{Binary}: \textbf{raw(max length)}, \textbf{varbinary(max length)}, \textbf{blob}, \textbf{bytea}.
\end{itemize}
CREATE TABLE语句的基本形式如下:
\begin{lstlisting} 
CREATE TABLE 表名(
    标识符 类型 (not null,如果不能留空) (primary key,隐式not null,不重复,只有一个),
    check(条件) ,
    标识符 类型 ,
    ……
    unique(标识符,标识符)
);
\end{lstlisting}
\paragraph{Constraint} 创建表的时候,我们可以添加一系列限制来规范输入:
\begin{itemize} 
    \item \textbf{not null}: 附加在标识符之后,表示填入数据列不得为null;
    \item \textbf{primary key}: 附加在标识符后,隐含not null和unique;单独成行可以标记基本键值元组;
    \item \textbf{check}: 添加在标识符后方,多个用逗号分隔: check(bool expr); 
    \item \textbf{unique}: 用逗号分隔,可以限制单个元素唯一,也可以限制某一个元组唯一;或者,直接加在标识符后方; 
    \item \textbf{constraint}: 若独立成行,可以标记外键(constraint 限制名 foreign key(外键名) references 表名(键名));
                               总之是用来命名“限制”的;
    \item \textbf{foreign key}: 标准语法是:foreign key 键或者元组 references 表名(键或者元组);
    \item \textbf{default}: 附加在键后面,表示若没有显式初始化,则直接初始化为此值;
    \item \textbf{references}: 附加在键后面可以表示参考的外部键值是什么,保持内外一致;
\end{itemize} 

\subsection{Null} 
Null指的是“不存在值”,与其它编程语言不同,不能够运算。

\subsection{ALTER}

\subsection{DROP}

\subsection{INSERT}

\section{DML语句}

\subsection{SELECT:查询} 
Select语句的语句结构如下: 
\begin{lstlisting} 
[WITH [RECURSIVE] with_query [, ...]]
SELECT [ALL | DISTINCT [ON (expression [, ...])]]
    * | expression [ [AS] output_name ] [, ...]
[FROM from_item [, ...]]
[WHERE condition]
[GROUP BY grouping_element [, ...]]
[HAVING condition]
[WINDOW window_name AS (window_definition) [, ...]]
[{UNION | INTERSECT | EXCEPT} [ALL | DISTINCT] select]
[ORDER BY expression [ASC | DESC | USING operator] [NULLS {FIRST | LAST}] [, ...]]
[LIMIT {count | ALL}]
[OFFSET start [ROW | ROWS]]
[FETCH {FIRST | NEXT} [count] {ROW | ROWS} ONLY]
[FOR {UPDATE | NO KEY UPDATE | SHARE | KEY SHARE} [OF table_name [, ...]] [NOWAIT | SKIP LOCKED] [...]]
\end{lstlisting}

\subsection{UPDATE} 

\subsection{DELETE}

\section{Schema}

\section{Window Function} 

\section{Trigger} 

\section{一些其它语句} 
\subsection{not in} 










\end{document}